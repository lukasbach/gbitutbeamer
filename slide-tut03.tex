\def\tutdate{11.11.2016}
\ifdefined\compileall \else
\ifdefined\compiletype
	\documentclass[handout]{beamer}
\else
	\documentclass{beamer}
	\def\compiletype{livebeamer}
\fi

\usepackage{templates/beamerthemekitwide}

\usepackage[utf8]{inputenc}
\usepackage[T1]{fontenc}
\usepackage[ngerman]{babel}
\usepackage{listings}
\usepackage{hyperref}
\usepackage{graphicx}

\usepackage{amsmath}
\usepackage{amsthm}
\usepackage{amssymb}
\usepackage{polynom}

\usepackage{ifthen}
\usepackage{adjustbox} % for \adjincludegraphics

\usepackage{tikz}
\usepackage{listings}

\usepackage[]{algorithm2e}

\usepackage{colortbl}
\usepackage{verbatim}
\usepackage{alltt}
\usepackage{changes}

\usepackage{pdfpages}
\usepackage{tabularx}

\usepackage{euler}

\newcommand{\markBlue}[1]{\textcolor{kit-blue100}{#1}}
\newcommand{\markGreen}[1]{\textcolor{kit-green100}{#1}}
\newcommand{\vertspace}{\vspace{.2cm}}

%\newcommand{\#}{\markBlue{#1}}

%\newcommand{\pitem}{\pause\item}
\newcommand{\p}{\pause}

% -- MATH MACROS
\newcommand{\thistheoremname}{}
\newcommand{\G}{\mathbb{Z}}
\newcommand{\B}{\mathbb{B}}
\newcommand{\R}{\mathbb{R}}
\newcommand{\N}{\mathbb{N}}
\newcommand{\Q}{\mathbb{Q}}
\newcommand{\C}{\mathbb{C}}
\newcommand{\Z}{\mathbb{Z}}
\newcommand{\F}{\mathbb{F}}
\newcommand{\mi}{\mathrm{i}}
\renewcommand{\epsilon}{\varepsilon}


\newenvironment<>{taskblock}[1]{%
	\setbeamercolor{block title}{fg=kit-orange15,bg=kit-orange100}
	\setbeamercolor{block body}{fg=black,bg=kit-orange30}%
	\begin{block}#2{#1}}{\end{block}}

\setbeamertemplate{enumerate items}[default]

% Aussagenlogik Symbole
\newcommand{\W}{w}
\renewcommand{\F}{f}

% Kodierung
\newcommand{\frepr}{\textbf{repr}}
\newcommand{\fRepr}{\textbf{Repr}}
\newcommand{\fZkpl}{\textbf{Zkpl}}
\newcommand{\fbin}{\textbf{bin}}
\newcommand{\fdiv}{\textbf{ div }}
\newcommand{\fmod}{\textbf{ mod }}

% Speicherabbild
\newenvironment{memory}{\begin{tabular}{r | l}Adresse&Wert\\\hline\hline}{\end{tabular}}
\newcommand{\memrow}[2]{#1 & #2 \\\hline}

% Praedikatenlogik
\newcommand{\objequiv}{\stackrel{\cdot}{=}}
\newcommand{\pval}{val_{D,I,\beta}}

% Neue Befehle
\newcommand{\ip}{\pause} % inline pause, für mitten im satz
\newcommand{\pitem}{\pause\item} % für aufzählungen
\newcommand{\bp}{\pause} % block pause, für zwischen blöcken

\title[Grundbegriffe der Informatik]{Grundbegriffe der Informatik\\Tutorium 33}
\subtitle{}
\author{Lukas Bach, lukas.bach@student.kit.edu}
\date{\tutdate}

\institute{}

\titlelogo{lukasbach}
\titleimage{bg}



\ifthenelse{\equal{\compiletype}{livebeamer}}
	{
		\def\livebeamermode{1}
	}{}

\ifthenelse{\equal{\compiletype}{print}}
	{
		\def\printmode{1}
	}{}

\setbeamercovered{invisible}

%\usepackage[citestyle=authoryear,bibstyle=numeric,hyperref,backend=biber]{biblatex}
%\addbibresource{templates/example.bib}
%\bibhang1em

\begin{document}
	
\selectlanguage{ngerman}


%title page
\begin{frame}
	\titlepage
\end{frame}

%table of contents
\ifdefined\printmode
	\ifdefined\compileall \else
	\begin{frame}{Gliederung}
		\tableofcontents
	\end{frame}
\fi\fi

\fi


\section{Aussagenlogik}

\begin{frame}{Aussagenlogik}
	\begin{itemize}
		\pitem Das wars erst mal zu formalen Sprachen.
		\pitem Heute ist Freitag.
		\pitem Die Menge aller Männer dieser Welt ist disjunkt zur Menge aller Frauen dieser Welt.
	\end{itemize}

	\pause
	
	Das sind alles Aussagen. \pause Aussagen sind entweder \emph{wahr} \pause oder \emph{falsch}.
\end{frame}

\begin{frame}{Aussagenlogik}
	\pause 
	
	Wir kapseln Aussagen und verwendet Variablen dafür. \pause 
	
	Zum Beispiel:
	
	\begin{itemize}
		\pitem $A := $ ``Die Straße ist nass.''
		\pitem $B := $ ``Es regnet.''
	\end{itemize}

	\pause Aussagen lassen sich verknüpfen:
	
	\begin{itemize}
		\pitem \markGreen{Logisches Und:} \pause $A \land B \pause = A$ und $B \pause = $ Die Straße ist nass und es regnet.
		\pitem \markGreen{Logisches Oder:} \pause $A \lor B \pause = A$ oder $B \pause = $ Die Straße ist nass oder es regnet\pause . Es kann auch beides wahr sein.
		\pitem \markGreen{Negierung:} \pause $\lnot A \pause = $ nicht $A \pause = $ Die Straße ist nicht nass.
		\pitem \markGreen{Implikation:} \pause $A \rightarrow B \pause = $ Aus $A$ folgt $B \pause = $ Wenn die Straße nass ist, dann regnet es.
		\pitem \markGreen{Äquivalenz:} \pause $A \leftrightarrow B \pause = A$ und $B$ sind äquivalent $\pause = $ Die Straße ist \emph{genau dann} nass, \emph{wenn} es regnet.
		\begin{itemize}
			\pitem $A \leftrightarrow B = (A \rightarrow B) \land (B \rightarrow A)$\pause , also die Straße ist nass wenn es regnet \emph{und} es regnet wenn die Straße nass ist.
		\end{itemize}
	\end{itemize}

\end{frame}

\begin{frame}{Übung zu Aussagenlogik}
	
	\begin{itemize}
		\item $A := $ ``Die Straße ist nass.''
		\item $B := $ ``Es regnet.''
		\item $C := $ ``$\pi$ ist gleich $3$.''
	\end{itemize}

	\begin{itemize}
		\pitem Was ist $B \rightarrow C$? \pause ``Wenn es regnet, ist $\pi$ gleich $3$.''
	\end{itemize}

	\pause
	
	% Aus Skriptum, Kapitel 5.3 Boolesche Funktionen
	\begin{center}
		\begin{tabular}{c|c||c|c|c|c}%*{2}{>{$}c<{$}}|*{4}{>{$}c<{$}}
			\hline
			$x_1$ & $x_2$ & $\lnot x_1$ & $x_1 \land x_2$ & $x_1 \lor x_2$ & $x_1 \rightarrow x_2$ \\
			\hline
			\F & \F & \W & \F & \F & \W \\
			\F & \W & \W & \F & \W & \W \\
			\W & \F & \F & \F & \W & \F \\
			\W & \W & \F & \W & \W & \W \\
			\hline
		\end{tabular}
	\end{center}
	
\end{frame}

\begin{frame}{Syntax der Aussagenlogik}
	\pause
	Menge der Aussagevariablen:
	
	\pause\quad $Var_{AL} \pause \subseteq \{P_i : i \in \N_0\}$ \pause oder $\{P, Q, R, S, \dots\}$
	
	\pause Alphabet der Aussagenlogik:
	
	\pause\quad $A_{AL} = \{(, ), \lnot, \land, \lor, \rightarrow, \leftrightarrow\} \cup Var_{AL}$
\end{frame}

\begin{frame}{Boolesche Funktionen}
\begin{block}{Boolesche Funktionen}
	Eine boolsche Funktion ist eine Abbildung \pause der Form $f: \B^n \rightarrow \B$ \pause mit $\B = \{w, f\}$.
\end{block}

Typische Boolsche Funktionen\pause : $b_\lnot (x) \pause = \lnot x$\pause , $b_\lor (x_1, x_2) \pause = x_1 \lor x_2$ \dots
\end{frame}

\begin{frame}{Interpretationen}
	\begin{block}{Interpretation}
		\pause Eine Interpretation ist eine Abbildung $I : V \rightarrow \B$\pause , die einer Variablenmenge eine ``Interpretation''\pause , also wahr oder falsch zuordnet.
	\end{block}

\pause Weiter legt man $val_I(F)$ als Auswertung einer aussagenlogischer Formel $F$ fest.
\pause
	\newcommand{\val}{val}
	\begin{align*}
	\val_I(X)         &= I(X) \\
	\val_I(\lnot G)   &= b_{\lnot}(\val_I(G)) \\
	\val_I(G \land H) &= b_{\land}(\val_I(G), \val_I(H)) \\
	\val_I(G \lor H)  &= b_{\lor}(\val_I(G)  \val_I(H)) \\
	\val_I(G \rightarrow H)&= b_{\rightarrow}(\val_I(G), \val_I(H))
	\end{align*}
\end{frame}

\begin{frame}{Übung zu Interpretationen}
	
	
	\begin{itemize}
		\item Wie viele Interpretationen gibt es bei k = 1, 2, 3 Variablen?
		\item Wie viele Interpretationen gibt es bei k+1 Variablen im Vergleich zu k Variablen?
	\end{itemize}
	
\end{frame}	

\begin{frame}{Übung zur Aussagenlogik}
	\pause Sei $A := \W, B := \W, C := \F$.
	
	\begin{itemize}
		\pitem Ist $(A \land B) \lor \lnot C$ wahr oder falsch? \pause $(A \land B) \lor \lnot C \pause = (\W \land \W) \lor \lnot \F \pause = \W \lor \lnot \F = \pause \W \lor \W \pause = \W$\pause , die Aussage ist also wahr.
		\pitem Ist $\lnot (A \lor A)$ wahr oder falsch? \pause Falsch! \pause Wann ist $\lnot (A \lor A)$ im allgemeinen wahr? \pause Genau dann, wenn $\lnot A$ wahr ist.
	\end{itemize}

	\pause

	\begin{block}{Aussagen Äquivalenz}
		Erinnerung: \pause $A \leftrightarrow B$ heißt: \pause $A \rightarrow B \land B \rightarrow A$. 
		
		\pause Wenn zwei Aussagen äquivalent sind, sind ihre Wahrheitswerte immer gleich\pause , wenn die Wahrheitswerte, von denen sie abhängen, gleich sind. 
		
		\pause Mann sagt und schreibt dann: \pause $A$ ist \emph{genau dann} wahr, \emph{wenn} $B$ wahr ist.
	\end{block}

	\begin{itemize}
		\pitem $\lnot (A \lor A)$ ist genau dann wahr\pause , wenn $\lnot A$ wahr ist\pause , also gilt:  $\lnot (A \lor A) \pause \leftrightarrow \lnot A$. 
	\end{itemize}
\end{frame}

\begin{frame}{Mehr zu Äquivalenz}
\pause
	\begin{block}{Alternative Definition zu Äquivalenz}
		Zwei Formeln G und H heißen äquivalent, wenn für jede Interpretation gilt $val_I(G) = val_I(H)$.
	\end{block}\pause

	Vorher Äquivalenz von Formeln unter gegebener Interpretation\pause , diesmal Äquivalenz von Formeln unter beliebiger Interpretation.\pause

	\textbf{Bemerkung}\\
	\begin{itemize}
		\pitem Man schreibt $G \equiv  H$
		\pitem $\mathbb{B}^V \rightarrow \mathbb{B}: I \mapsto val_I(G)$
	\end{itemize}\pause
	\textbf{Beispiele}\\\pause
	$(\lnot(\lnot P))$ ist äquivalent zu $P$\\\pause
	$(\lnot(P\land Q))$ ist äquivalent zu $((\lnot P) \lor (\lnot Q))$
\end{frame}

\begin{frame}{Beispiele zu Äquivalenz}
	\begin{itemize}
		\pitem Ein Wort $w$ hat die Länge $n$ $\leftrightarrow |w| = n$.
		\pitem Die Vereinigung zweier Mengen $A$ und $B$ hat die Kardinalität $|A| + |B|$ \pause $\leftrightarrow$ $A \cap B = \emptyset$ \pause $\leftrightarrow$ $A$ und $B$ sind disjunkt.
		\pitem $p$ ist eine rationale Zahl \pause $\leftrightarrow$ $p$ lässt sich darstellen als $p = \frac{a}{b}, a\in \Z, b \in \N$ \pause $\leftrightarrow$ $p \in \Q$.
	\end{itemize}	
\end{frame}

\begin{frame}{Wahrheitstabellen}
	\begin{itemize}
		\item $(((P \rightarrow Q) \lor Q) \rightarrow (P \land \lnot Q))$
	\end{itemize}

	\begin{center}
		\begin{tabular}{c|c||c|c|c|c}%*{2}{>{$}c<{$}}|*{4}{>{$}c<{$}}
			\hline
				$P$ & $Q$ & $(P \land Q)$ & $\lor Q$ & $\rightarrow$ & $(P \land \lnot Q)$ \\\hline
				
				\visible<1->{\W} & \visible<1->{\W} & \visible<2->{\W} & \visible<6->{\W} & \visible<14->{\F} & \visible<10->{\F} \\\hline
				
				\visible<1->{\W} & \visible<1->{\F} & \visible<3->{\F} & \visible<7->{\F} & \visible<15->{\W} & \visible<11->{\W} \\\hline
				
				\visible<1->{\F} & \visible<1->{\W} & \visible<4->{\F} & \visible<8->{\W} & \visible<16->{\F} & \visible<12->{\F} \\\hline
				
				\visible<1->{\F} & \visible<1->{\F} & \visible<5->{\F} & \visible<9->{\F} & \visible<17->{\W} & \visible<13->{\F} \\\hline
				
		\end{tabular}
	\end{center}
\end{frame}

\begin{frame}{Übungen zu Aussagenlogik}
	\begin{taskblock}{Übungen zu Aussagenlogik}
		\begin{itemize}
			\item Schreibe Wahrheitstabellen zu den Formeln um den Wahrheitsgehalt festzustellen.
				\item $\lnot(P \land Q) \land \lnot (Q \land P)$
				\item $(P \land Q \land R) \leftrightarrow (\lnot P \lor Q)$
				\item $(A\land(B\lor C))\leftrightarrow ((A\land B)\lor(A\land C))$
			\item Welche dieser Aussagen sind wahr?
				\item $\lnot (P \land Q) = \lnot P \lor \lnot Q$
				\item $P \land P = P \lor P$
				\item $(P \lor Q) \land R = (P \land R) \lor (Q \land R)$
		\end{itemize}
	\end{taskblock}
\end{frame}

\begin{frame} {Wahrheitstabellen}
	\begin{center}
		\begin{tabular}{|c|c|c|c|c|c|c|}
			\hline
			$A$&$B$& $\lnot A$& $A \land B$ & $A\lor B$ &$A\rightarrow B$ &$A \leftrightarrow B$\\
			\hline
			w&w&f&w&w&w&w\\
			w&f&f&f&w&f&f\\
			f&w&w&f&w&w&f\\
			f&f&w&f&f&w&w\\
			\hline
		\end{tabular}
	\end{center}

	\begin{taskblock}{Aufgabe}
		Finde einen logischen Ausdruck in A und B unter Verwendung von $\land, \lor$ und $\lnot$, der die Aussage ``Entweder A oder B'' repräsentiert	
	\end{taskblock}
\end{frame}

\begin{frame}{Wahrheitstabellen}
	
	\begin{taskblock}{Aufgabe}
		Finde einen logischen Ausdruck in A und B unter Verwendung von $\land, \lor$ und $\lnot$, der die Aussage ``Entweder A oder B'' repräsentiert	
	\end{taskblock}

	\textbf{Lösung}
	\begin{center}
		\begin{tabular}{|c|c|c|c|c|}
			\hline
			$A$&$B$& $A \land \lnot B$& $\lnot A \land B$ & $(A \land \lnot B) \lor (\lnot A \land B) $\\
			\hline
			w&w&f&f&f\\
			w&f&w&f&w\\
			f&w&f&w&w\\
			f&f&f&f&f\\
			\hline
		\end{tabular}
	\end{center}
\end{frame}


\begin{frame}{Weitere Begriffe}\pause
	\begin{block}{Tautologie}\pause
		Die Formel $G$ ist eine Tautologie (oder allgemeingültig)\pause , wenn $G$ für alle Interpretationen wahr ist.
	\end{block}\pause
	\begin{block}{Erfüllbarkeit}\pause
		Eine Formel $G$ ist erfüllbar\pause , wenn sie für mindestens eine Interpretation wahr ist.
	\end{block}
	\pause
	\begin{block}{Lemma}
		Wenn $G\equiv H$ ist, dann ist $G \leftrightarrow H$ eine Tautologie.
	\end{block}
\end{frame}

\begin{frame} {Übung zu Tautologien}
Sind das Tautologien?
\begin{itemize}
	\item $(G \rightarrow (H \rightarrow K)) \rightarrow ((G \rightarrow H) \rightarrow (G \rightarrow K))$ \pause \hspace{0.3cm} Ja
	\item $(\lnot P \rightarrow Q) \land R \lor P$ \pause \hspace{0.3cm} Nein
	\item $G \rightarrow (H \rightarrow G)$ \pause \hspace{0.3cm} Ja
	\item $(\lnot P \rightarrow \lnot Q) \rightarrow ((\lnot P \rightarrow Q) \rightarrow P)$ \pause \hspace{0.3cm} Ja
\end{itemize}
\end{frame}


\begin{frame} {Übung zu Erfüllbarkeit}
	Sind die folgenden Ausdrücke erfüllbar?
	\begin{itemize}
		\item $ \lnot (A \lor \lnot A) $ \pause \hspace{0.3cm} nein
		\item $(P \land \lnot Q) \lor (\lnot P \land R)$ \pause \hspace{0.3cm} Ja
		
	\end{itemize}
\end{frame}

\ifdefined\compileall
\else


\ifthenelse{\equal{\compiletype}{print}}
{

\begin{frame}{Informationen}
	
	\begin{columns}
		\begin{column}{0.5\textwidth}
			
			\begin{block}{Zum Tutorium}
				\begin{itemize}
					\item Lukas Bach
					\item Tutorienfolien auf: 
					\begin{itemize}
						\item \url{http://gbi.lukasbach.com}
					\end{itemize}
					\item Tutorium findet statt:
					\begin{itemize}
						\item Donnerstags, 14:00 - 15:30
						\item 50.34 Informatikbau, -107
					\end{itemize}
				\end{itemize}
			\end{block}
			
			\begin{block}{Mehr Material}
				\begin{itemize}
					\item ILIAS der Vorlesung:
					\begin{itemize}
						\item kommt noch. %TODO
					\end{itemize}
					\item Ehemalige GBI Webseite:
					\begin{itemize}
						\item \url{http://gbi.ira.uka.de}
						\item Altklausuren!
					\end{itemize}
				\end{itemize}
			\end{block}
			
		\end{column}
		\begin{column}{0.5\textwidth}
			
			\begin{block}{Zur Veranstaltung}
				\begin{itemize}
					\item Grundbegriffe der Informatik
					\item Klausurtermin:
					\begin{itemize}
						\item 06.03.2017, 11:00
						\item Zwei Stunden Bearbeitungszeit
						\item 6 ECTS für Informatiker und Informationswirte, 4 ECTS für Mathematiker und Physiker
					\end{itemize}
				\end{itemize}
			\end{block}
			
			\begin{block}{Zum Übungsschein}
				\begin{itemize}
					\item Übungsblatt jede Woche
					\item Ab 50\% insgesamt hat man den Übungsschein
					\item Keine Voraussetzung für die Klausur, aber für das Modul
				\end{itemize}
			\end{block}
			
		\end{column}
	\end{columns}
	
\end{frame}

}{}

\ifdefined\livebeamermode
	\begin{frame}
		\includegraphics[width=\linewidth]{images/thatsall.png}
	\end{frame}
\fi

\end{document}

\fi