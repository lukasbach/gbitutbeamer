\def\tutdate{15.12.2016}
\ifdefined\compileall \else
\ifdefined\compiletype
	\documentclass[handout]{beamer}
\else
	\documentclass{beamer}
	\def\compiletype{livebeamer}
\fi

\usepackage{templates/beamerthemekitwide}

\usepackage[utf8]{inputenc}
\usepackage[T1]{fontenc}
\usepackage[ngerman]{babel}
\usepackage{listings}
\usepackage{hyperref}
\usepackage{graphicx}

\usepackage{amsmath}
\usepackage{amsthm}
\usepackage{amssymb}
\usepackage{polynom}

\usepackage{ifthen}
\usepackage{adjustbox} % for \adjincludegraphics

\usepackage{tikz}
\usepackage{listings}

\usepackage[]{algorithm2e}

\usepackage{colortbl}
\usepackage{verbatim}
\usepackage{alltt}
\usepackage{changes}

\usepackage{pdfpages}
\usepackage{tabularx}

\usepackage{euler}

\newcommand{\markBlue}[1]{\textcolor{kit-blue100}{#1}}
\newcommand{\markGreen}[1]{\textcolor{kit-green100}{#1}}
\newcommand{\vertspace}{\vspace{.2cm}}

%\newcommand{\#}{\markBlue{#1}}

%\newcommand{\pitem}{\pause\item}
\newcommand{\p}{\pause}

% -- MATH MACROS
\newcommand{\thistheoremname}{}
\newcommand{\G}{\mathbb{Z}}
\newcommand{\B}{\mathbb{B}}
\newcommand{\R}{\mathbb{R}}
\newcommand{\N}{\mathbb{N}}
\newcommand{\Q}{\mathbb{Q}}
\newcommand{\C}{\mathbb{C}}
\newcommand{\Z}{\mathbb{Z}}
\newcommand{\F}{\mathbb{F}}
\newcommand{\mi}{\mathrm{i}}
\renewcommand{\epsilon}{\varepsilon}


\newenvironment<>{taskblock}[1]{%
	\setbeamercolor{block title}{fg=kit-orange15,bg=kit-orange100}
	\setbeamercolor{block body}{fg=black,bg=kit-orange30}%
	\begin{block}#2{#1}}{\end{block}}

\setbeamertemplate{enumerate items}[default]

% Aussagenlogik Symbole
\newcommand{\W}{w}
\renewcommand{\F}{f}

% Kodierung
\newcommand{\frepr}{\textbf{repr}}
\newcommand{\fRepr}{\textbf{Repr}}
\newcommand{\fZkpl}{\textbf{Zkpl}}
\newcommand{\fbin}{\textbf{bin}}
\newcommand{\fdiv}{\textbf{ div }}
\newcommand{\fmod}{\textbf{ mod }}

% Speicherabbild
\newenvironment{memory}{\begin{tabular}{r | l}Adresse&Wert\\\hline\hline}{\end{tabular}}
\newcommand{\memrow}[2]{#1 & #2 \\\hline}

% Praedikatenlogik
\newcommand{\objequiv}{\stackrel{\cdot}{=}}
\newcommand{\pval}{val_{D,I,\beta}}

% Neue Befehle
\newcommand{\ip}{\pause} % inline pause, für mitten im satz
\newcommand{\pitem}{\pause\item} % für aufzählungen
\newcommand{\bp}{\pause} % block pause, für zwischen blöcken

\title[Grundbegriffe der Informatik]{Grundbegriffe der Informatik\\Tutorium 33}
\subtitle{}
\author{Lukas Bach, lukas.bach@student.kit.edu}
\date{\tutdate}

\institute{}

\titlelogo{lukasbach}
\titleimage{bg}



\ifthenelse{\equal{\compiletype}{livebeamer}}
	{
		\def\livebeamermode{1}
	}{}

\ifthenelse{\equal{\compiletype}{print}}
	{
		\def\printmode{1}
	}{}

\setbeamercovered{invisible}

%\usepackage[citestyle=authoryear,bibstyle=numeric,hyperref,backend=biber]{biblatex}
%\addbibresource{templates/example.bib}
%\bibhang1em

\begin{document}
	
\selectlanguage{ngerman}


%title page
\begin{frame}
	\titlepage
\end{frame}

%table of contents
\ifdefined\printmode
	\ifdefined\compileall \else
	\begin{frame}{Gliederung}
		\tableofcontents
	\end{frame}
\fi\fi

\fi

% pop quiz

\section{Prädikatenlogik}
\begin{frame}{Grundlagen zu Prädikatenlogik}
	Prädikatenlogik (PL) \ip \markBlue{erweitert} Aussagenlogik durch Ergänzen von ``Prädikaten''\ip , einer Art von Funktionen, die Wahrheitswerte zurückgeben.
	
	\bp
	
	Alphabet der Prädikatenlogik:
	
	\begin{itemize}
		\pitem $\lnot, \land, \lor, \rightarrow, \leftrightarrow, (, )$, also Alphabet der Aussagenlogik.
		\pitem $\forall$ Allquantor \ip ($\forall x$ heißt ``für alle $x$ gilt...)
		\pitem $\exists$ Existenzquantor \ip ($\exists x$ heißt ``es existiert min. ein $x$... für das gilt...)
		\pitem $x,y,z,x_i \in Var_{PL}$ Variablen
		\pitem $c, d, c_i \in Const_{PL}$ Konstanten
		\pitem $f, g, h, f_i \in Fun_{PL}$ Funktionen %TODO Selligkeit ar(f_i) = anzahl parameter
		\pitem $R, S, R_i \in Rel_{PL}$ Relationen (funktionieren ähnlich wie Funktionen)
		\pitem $\objequiv$ Objektgleichheit
		\pitem $,$ Komma
	\end{itemize}
\end{frame}

\begin{frame}{Gliederung der Prädikatenlogik}
	\begin{block}{Terme}
		Ein Term ist ein Element aus der Sprache über $A_{Ter} := \{\markBlue{(}, \markBlue{)}, \markBlue{,}\} \cup Var_{PL} \cup Const_{PL} \cup Fun_{PL}$.
	\end{block}

	\bp
	
	\begin{block}{Atomare Formeln}
		Atomare Formeln sind zum Beispiel
		\begin{itemize}
			\pitem Objektgleichheiten $f_1 \objequiv f_2$
			\pitem Relation von Termen $R(t_1, t_2, ...)$
		\end{itemize}
	\end{block}

	\begin{block}{Stelligkeit einer Funktion}
		Die Stelligkeit $ar(f) \in \N_+$ einer Funktion gibt die Anzahl der Parameter von $f$ an. \ip (Analog Stelligkeit von Relationen $ar(R)$)
	\end{block}
	
	%\bp
\end{frame}

\begin{frame}{Verständnis von Termen, Atomaren Formeln, Stelligkeit}
\begin{itemize}
	\item Woraus kann ein Term bestehen? \pause
	\item[$\rightarrow$] Aus Klammern $\markBlue{(}, \markBlue{)}$, Kommas $\markBlue{,}$, Variablen, Konstanten, Funktionen.\pause
	\item Was davon sind atomare Formeln: $R(x) \land S(f(x, c))$, $R(x, g(c, f(y, x))$?\pause
	\item[$\rightarrow$] Nein, ja.\pause
	\item Was sind die Stelligkeiten folgender Funktionen: $f(a, b, c), g(a), h(a, b)$? \pause \item[$\rightarrow$] $4,1,2$.
\end{itemize}	
\end{frame}

\begin{frame}{Grammatik der Prädikatenlogik}
	Prädikatenlogische Formeln werden durch die Grammatik $G := (N_{Ter}, A_{Ter}, T, P_{Ter})$ erzeugt mit:
	
	\bp
	
	\begin{itemize}
		\pitem $m+1$ Nichtterminalsymbolen $N_{Ter} := \{T\} \cup \{L_i | i \in \N_+ \text{ und } i \leq m \}$ ($m = $ Maximale Stelligkeit von Funktionen)
		\pitem Terminalsymbolen: Alphabet, aus dem Terme erzeugbar sind
		%\pitem Startsymbol $T$
		\pitem Produktionen
		\begin{alignat*}{2}
		L_{i+1} &\to L_i , T &\qquad& \text{für jedes } i\in\N_+\text{ mit } i<m   \\
		L_1  &\to T \\ % ACHTUNG Komma ist hier ein Terminalsymbol, kein Trennsymbol
		T &\to c_i && \text{für jedes } c_i\in Const_{PL}\\
		T &\to x_i && \text{für jedes } x_i\in Var_{PL}\\
		T &\to f_i(L_{ar(f_i)} ) && \text{für jedes } f_i\in Fun_{PL}
		\end{alignat*}
	\end{itemize}

	\bp
	
	Beispiel: Seien Funktionen $f,g$ mit $ar(f) = 2, ar(g) = 1$, Konstante $c$ und Variablen $x,y$ gegeben. Was kann man damit machen?
\end{frame}

\begin{frame}{Grammatik der Prädikatenlogik}
	
	
	\begin{columns}
		\begin{column}{0.6\textwidth}
			Beispiel: Seien Funktionen $f,g$ mit $ar(f) = 2, ar(g) = 1$, Konstante $c$ und Variablen $x,y$ gegeben. Was kann man damit machen?\vspace{.2cm}
			
			\bp
			
			Dann: 
			\begin{itemize}
				\item $N_{Ter}=\{ T, L_1, L_2 \}$
				\item $\begin{aligned}[t]
				P_{Ter} = \{  L_2 & \to L_1 , T \\% ACHTUNG Komma ist hier ein Terminalsymbol, kein Trennsymbol
				L_1 & \to T  \\
				T   & \to c \\
				T   & \to x \\
				T   & \to y \\
				T   & \to g ( L_1 ) \\
				T   & \to f ( L_2 ) \}
				\end{aligned}
				$
			\end{itemize}
		\end{column}
		
		\begin{column}{0.4\textwidth}
			\bp
			
			\begin{taskblock}{Aufgabe zu Grammatiken und Prädikatenlogik}
				Welche dieser Formeln entsprechen dieser Grammatik?
				\vspace{.2cm}
				\begin{itemize}
					\pitem $f(c, g(x))$
					\pitem $f(x, y, c)$
					\pitem $g(f(c, c))$
					\pitem $g(g(f(g(x), g(f(c, c))))$
					\pitem $g(c\ip, f)c)$
				\end{itemize}
			
				\ip Bilde die Ableitungsbäume zu den korrekten Formeln.
			\end{taskblock}
		\end{column}
	\end{columns}
\end{frame}

\begin{frame}{Bindungsstärken}
	\begin{block}{Bindungsstärke}
		Verschiedene Operanden ``binden'' stärker als andere. \ip Wenn ein Operand stärker als die umliegenden Operanden bindet, tritt derselbe Effekt auf, wie wenn Klammerung geschehen würde.
	\end{block}

	\bp
	
	Bindungsstärken absteigen:
	\begin{itemize}
		\ip \item $\forall / \exists\ip,\lnot\ip,\land\ip,\lor\ip,\rightarrow/\leftarrow\ip,\leftrightarrow$
	\end{itemize}

	\bp
	
	Finde äquivalente Formeln, die mit möglichst wenig Klammern auskommen:
	\begin{itemize}
		\pitem $\exists x \forall y (R(f(x), g(x)) \lor \forall z R(c, x)$ %TODO finish
	\end{itemize}

\end{frame}

\begin{frame}{Semantik von prädikatenlogischen Formeln}
	Um prädikatenlogische Formeln zu interpretieren, brauchen wir einige neue Mengen:
	
	\begin{itemize}
		\pitem Interpretation $(D, I)$\ip, bestehend aus...
		\begin{itemize}
			\pitem Universum $D \neq \emptyset$ mit...
			\begin{itemize}
				\pitem $I(c_i) \in D$ für $c_i \in Const_{PL}$
				\pitem $I(f_i) : D^{ar(f_i)} \rightarrow D$ für $f_i \in Fun_{PL}$
				\pitem $I(R_i) \subseteq D^{ar(R_i)}$ für $R_i \in Rel_{PL}$
				\pitem $I$ bildet weißt also den Komponenten Bedeutungen zu, ``definiert diese''
			\end{itemize}
		
			\pitem Variablenbelegung $\beta : Var_{PL} \rightarrow D$, z.B. $\beta(x) := 3, \beta(y) := 11$
			\begin{itemize}
				\pitem $\beta$ definiert also Variablenwerte
			\end{itemize}
		\end{itemize}
	
		\bp
	
		\item Damit können wir prädikatenlogische Formeln definieren!
	\end{itemize}

	\p
	
	\begin{block}{$\pval$}
		Die Funktion $\pval : L_{Ter} \cup L_{For} \rightarrow D \cup \B$ \ip weißt einer prädikatenlogischen Formel eine Bedeutung \ip (Wahrheitsgehalt für Formeln und Element des Universums für Terme) zu.
	\end{block}
\end{frame}

\begin{frame}{Quantoren}
\begin{itemize}
	\pitem $\forall x p(x)$ heißt\ip: für alle $x \in D$ gilt die Aussage $p(x)$.
	\pitem $\exists x p(x)$ heißt\ip: für (mindestens) ein $x \in D$ gilt die Aussage $p(x)$.
	\pitem Gilt $\forall x \exists y \quad p(x,y) = \exists y \forall x \quad p(x,y)$?
	\begin{itemize}
		\pause\item Zum Beispiel $p(x,y) := $``Person $x$ ist mit Person $y$ verheiratet.
		\pause\item Also:
		\begin{itemize}
			\pitem $\forall x \exists y \quad p(x,y) = $ Für jede Person $x$ gibt es eine Person $y$, mit der $x$ verheiratet ist.
			\pitem $\exists y \forall x \quad p(x,y) = $ Es gibt eine Person $y$, sodass für alle Personen $x$ gilt, dass $x$ mit allen Personen $y$ verheiratet ist.
		\end{itemize}
		\pitem Eher nicht. \ip Reihenfolge ist also wichtig!
	\end{itemize}
\end{itemize}
\end{frame}

\begin{frame}{Bindungsbereich von Quantoren}
	Quantoren binden Variablen nur zu der zugehörigen Teilformel.
	
	\bp
	
	\begin{itemize}
		\item Zum Beispiel: $p(x) \land \markBlue{\forall x \exists y} \markGreen{(p(x) \land q(x,y,z)} \rightarrow r(x) $
		\pitem Welcher Teil der Formel muss für alle $x$ gelten? Welcher für alle $y$?
		\pitem Variablen, die nicht im Wirkungsbereich eines Quantors liegen, nennt man \markGreen{frei}.
	\end{itemize}

	\bp Überschattung ist möglich\ip, daher durch Quantoren definierte Variablen beziehen sich immer auf den \markBlue{nächsten} Quantor.
	\begin{itemize}
		\pitem Ist $\forall x (p(x) \land \forall x (\lnot p(x))))$ erfüllbar?
		\pause\item Ja: $\forall x (p(x) \land \forall \hat{x} (\lnot p(\hat{x}))))$ 
	\end{itemize}
\end{frame}

\begin{frame}{Bindungsbereich von Quantoren}
	Substitution ist möglich.\ip Dabei wird eine \markBlue{freie} Variable durch einen Term ersetzt, die Substitution wird mit $\beta[a/b]$ bezeichnet, wobei $a$ durch $b$ ersetzt wird.
	
	\vertspace
	
	%TODO alternative Schreibweise für Substitution: \sigma_{x/y}
	
	\bp
	
	Führe die folgenden Subsitutionen durch:
	\begin{itemize}
		\pitem $\beta[x/5] (p(x) \lor q(x,y)) \pause = p(5) \lor q(5,y)$
		\pitem $\beta[x/5] (p(x) \lor \forall x (q(x,y)) \pause = p(5) \lor \forall x (q(x,y))$
		\pitem $\beta[x/y, y/x, z/f(z)] (p(x) \land q(x,y)) \pause = p(f(z)) \land q(y,x)$
	\end{itemize}

	\vertspace
	
	\bp
	
	Welche der Variablen sind gebunden (und durch welche Quantoren), welche sind frei? 
	\begin{itemize}
		\pitem $p(x) \rightarrow \forall x \exists y (p(x) \land q(y,z) \leftrightarrow \forall z (q(x,z)))$
		\pitem $\forall y(p(f(x,y))) \lor \exists z(q(z,f(y,z)))$
	\end{itemize}

	%TODO kollisionsfreiheit fehlt noch, evtl daran erklären
\end{frame}

\begin{frame}{Beispiel zur Semantik}
	Unterschied zwischen $\pval$ und $I$? \pause $I$ ist für Einzelteilen (Konstanten, Funktionen, Relationen) eine Bedeutung zuweist, und $\pval$ einer ganzen Formel eine Bedeutung zuweist.\pause\vertspace
	
	Beispiel:\vertspace\ip
	
	Sei $D := \N_0, I(c) := 10, ar(f) := 2, ar(p) := 1, ar(q) := 2, \beta(x) := 7$.\ip
	
	Sei $I(f) : \N_0^2 \rightarrow \N_0, (x,y) \mapsto x-y$.\ip
	
	Sei $ar(R) := 2, I(R) := \{(x,y) | x \leq y\}$.\ip
	
	Sei $I(p(x)) = w :\Leftrightarrow x \geq 5, I(q(x,y)) = w :\Leftrightarrow x \geq y$.
\end{frame}

\begin{frame}{Beispiel zur Semantik}
	
	Sei $D := \N_0, I(c) := 10, ar(f) := 2, ar(p) := 1, ar(q) := 2, \beta(x) := 7$.
	
	Sei $I(f) : \N_0^2 \rightarrow \N_0, (x,y) \mapsto x-y$.
	
	Sei $ar(R) := 2, I(R) := \{(x,y) | x \leq y\}$.
	
	Sei $I(p(x)) = w :\Leftrightarrow x \geq 5, I(q(x,y)) = w :\Leftrightarrow x \geq y$.
	
	\begin{itemize}
		\item $T_1 := p(x) \rightarrow \exists y (q(y,x) \land p(y))$, was ist $\pval(T_1)$?
		
		\pause
		\begin{itemize}
			\item Wähle $y = 8 \in \N_0$. \ip Dann: $I(q(8,7)) = w\ip, I(p(8)) = w$\ip, also $\pval(\exists y (q(y,x) \land p(y))) = w$\ip, und $\pval(T_1) = w$.
		\end{itemize}
		
		\pause
		
		\item $T_2 := p(x) \rightarrow \exists y (q(f(c,y), x) \land p(y))$, was ist $\pval(T_2)$?
		
		\pause
		\begin{itemize}
			\item $\pval(p(x)) = w$
			\ip\item $\pval(q(f(c,y), x)) \ip = \pval(q(f(10,y), x))\ip = \pval(q(10-y, 7))\ip = w$ für $y \in \{0,1,2\}$.
			\ip\item $\pval(p(y)) = w$ für $y \geq 5$.
			\ip\item Also: $\pval(T_2) = f$.
		\end{itemize}
	\end{itemize}
\end{frame}

\begin{frame}{Aufgaben zu Prädikatenlogik}
	\begin{taskblock}{Aufgaben zu Prädikatenlogik}
		Gegeben sind folgende Prädikate:
		\begin{itemize}
			\item $Vater(x,y) := $ wahr, gdw. $x$ Vater von $y$ ist, analog $Mutter(x,y)$.
			\item $M"annlich(x,y) := $ wahr, gdw. $x$ männlich ist, analog $Weiblich(x)$.
			\item $Verheiratet(x,y) := $ wahr, gdw. $x$ und $y$ verheiratet sind.
		\end{itemize}
	
	
		Drücke die folgenden Aussagen mit prädikatenlogischen Formeln aus:
		
		\begin{itemize}
			\pitem Jede männliche Person hat eine Mutter.
			\begin{itemize}
				\pause\item $\forall x \exists y (M"annlich(x) \rightarrow Mutter(y,x))$
				\pause\item Kann eine Person jetzt auch mehr als eine Mutter haben? \pause Widerspricht das der ursprünglichen Aussage?
			\end{itemize}
			\pitem Jeder Mann hat Kinder (plural!).
			\begin{itemize}
				\pause\item $\forall x \exists y \exists z (M"annlich(x) \rightarrow ( Vater(x,y) \land Vater(x,z) \land \lnot (y \objequiv z)))$
			\end{itemize}
		\end{itemize}
	\end{taskblock}
\end{frame}

\begin{frame}{Aufgaben zu Prädikatenlogik}
	\begin{taskblock}{Aufgaben zu Prädikatenlogik}
		Gegeben sind folgende Prädikate:
		\begin{itemize}
			\item $Vater(x,y) := $ wahr, gdw. $x$ Vater von $y$ ist, analog $Mutter(x,y)$.
			\item $M"annlich(x,y) := $ wahr, gdw. $x$ männlich ist, analog $Weiblich(x)$.
			\item $Verheiratet(x,y) := $ wahr, gdw. $x$ und $y$ verheiratet sind.
		\end{itemize}
		
		
		Drücke die folgenden Aussagen mit prädikatenlogischen Formeln aus:
		
		\begin{itemize}
			\pitem Jede Frau ist mit höchstens einem Mann verheiratet.
			\begin{itemize}
				\pause\item $\forall x \forall y \forall z (Weiblich(x) \land((M"annlich(y) \land M"annlich(z) \land \lnot (y \objequiv z) \land Verheiratet(x,y)) \rightarrow \lnot Verheiratet(x,z)))$
			\end{itemize}
			\pitem Wer männlich ist, ist nicht weiblich und umgekehrt.
			\begin{itemize}
				\pause\item $\forall x (M"annlich(x) \rightarrow \lnot Weiblich(x) \land Weiblich(x) \rightarrow \lnot M"annlich(x))$
			\end{itemize}
		\end{itemize}
	\end{taskblock}
\end{frame}

\ifdefined\compileall
\else


\ifthenelse{\equal{\compiletype}{print}}
{

\begin{frame}{Informationen}
	
	\begin{columns}
		\begin{column}{0.5\textwidth}
			
			\begin{block}{Zum Tutorium}
				\begin{itemize}
					\item Lukas Bach
					\item Tutorienfolien auf: 
					\begin{itemize}
						\item \url{http://gbi.lukasbach.com}
					\end{itemize}
					\item Tutorium findet statt:
					\begin{itemize}
						\item Donnerstags, 14:00 - 15:30
						\item 50.34 Informatikbau, -107
					\end{itemize}
				\end{itemize}
			\end{block}
			
			\begin{block}{Mehr Material}
				\begin{itemize}
					\item ILIAS der Vorlesung:
					\begin{itemize}
						\item kommt noch. %TODO
					\end{itemize}
					\item Ehemalige GBI Webseite:
					\begin{itemize}
						\item \url{http://gbi.ira.uka.de}
						\item Altklausuren!
					\end{itemize}
				\end{itemize}
			\end{block}
			
		\end{column}
		\begin{column}{0.5\textwidth}
			
			\begin{block}{Zur Veranstaltung}
				\begin{itemize}
					\item Grundbegriffe der Informatik
					\item Klausurtermin:
					\begin{itemize}
						\item 06.03.2017, 11:00
						\item Zwei Stunden Bearbeitungszeit
						\item 6 ECTS für Informatiker und Informationswirte, 4 ECTS für Mathematiker und Physiker
					\end{itemize}
				\end{itemize}
			\end{block}
			
			\begin{block}{Zum Übungsschein}
				\begin{itemize}
					\item Übungsblatt jede Woche
					\item Ab 50\% insgesamt hat man den Übungsschein
					\item Keine Voraussetzung für die Klausur, aber für das Modul
				\end{itemize}
			\end{block}
			
		\end{column}
	\end{columns}
	
\end{frame}

}{}

\ifdefined\livebeamermode
	\begin{frame}
		\includegraphics[width=\linewidth]{images/thatsall.png}
	\end{frame}
\fi

\end{document}

\fi