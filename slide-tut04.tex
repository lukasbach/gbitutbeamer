\ifdefined\compileall \else
\ifdefined\compiletype
	\documentclass[handout]{beamer}
\else
	\documentclass{beamer}
	\def\compiletype{livebeamer}
\fi

\usepackage{templates/beamerthemekitwide}

\usepackage[utf8]{inputenc}
\usepackage[T1]{fontenc}
\usepackage[ngerman]{babel}
\usepackage{listings}
\usepackage{hyperref}
\usepackage{graphicx}

\usepackage{amsmath}
\usepackage{amsthm}
\usepackage{amssymb}
\usepackage{polynom}

\usepackage{ifthen}
\usepackage{adjustbox} % for \adjincludegraphics

\usepackage{tikz}
\usepackage{listings}

\usepackage[]{algorithm2e}

\usepackage{colortbl}
\usepackage{verbatim}
\usepackage{alltt}
\usepackage{changes}

\usepackage{pdfpages}
\usepackage{tabularx}

\usepackage{euler}

\newcommand{\markBlue}[1]{\textcolor{kit-blue100}{#1}}
\newcommand{\markGreen}[1]{\textcolor{kit-green100}{#1}}
\newcommand{\vertspace}{\vspace{.2cm}}

%\newcommand{\#}{\markBlue{#1}}

%\newcommand{\pitem}{\pause\item}
\newcommand{\p}{\pause}

% -- MATH MACROS
\newcommand{\thistheoremname}{}
\newcommand{\G}{\mathbb{Z}}
\newcommand{\B}{\mathbb{B}}
\newcommand{\R}{\mathbb{R}}
\newcommand{\N}{\mathbb{N}}
\newcommand{\Q}{\mathbb{Q}}
\newcommand{\C}{\mathbb{C}}
\newcommand{\Z}{\mathbb{Z}}
\newcommand{\F}{\mathbb{F}}
\newcommand{\mi}{\mathrm{i}}
\renewcommand{\epsilon}{\varepsilon}


\newenvironment<>{taskblock}[1]{%
	\setbeamercolor{block title}{fg=kit-orange15,bg=kit-orange100}
	\setbeamercolor{block body}{fg=black,bg=kit-orange30}%
	\begin{block}#2{#1}}{\end{block}}

\setbeamertemplate{enumerate items}[default]

% Aussagenlogik Symbole
\newcommand{\W}{w}
\renewcommand{\F}{f}

% Kodierung
\newcommand{\frepr}{\textbf{repr}}
\newcommand{\fRepr}{\textbf{Repr}}
\newcommand{\fZkpl}{\textbf{Zkpl}}
\newcommand{\fbin}{\textbf{bin}}
\newcommand{\fdiv}{\textbf{ div }}
\newcommand{\fmod}{\textbf{ mod }}

% Speicherabbild
\newenvironment{memory}{\begin{tabular}{r | l}Adresse&Wert\\\hline\hline}{\end{tabular}}
\newcommand{\memrow}[2]{#1 & #2 \\\hline}

% Praedikatenlogik
\newcommand{\objequiv}{\stackrel{\cdot}{=}}
\newcommand{\pval}{val_{D,I,\beta}}

% Neue Befehle
\newcommand{\ip}{\pause} % inline pause, für mitten im satz
\newcommand{\pitem}{\pause\item} % für aufzählungen
\newcommand{\bp}{\pause} % block pause, für zwischen blöcken

\title[Grundbegriffe der Informatik]{Grundbegriffe der Informatik\\Tutorium 33}
\subtitle{}
\author{Lukas Bach, lukas.bach@student.kit.edu}
\date{\tutdate}

\institute{}

\titlelogo{lukasbach}
\titleimage{bg}



\ifthenelse{\equal{\compiletype}{livebeamer}}
	{
		\def\livebeamermode{1}
	}{}

\ifthenelse{\equal{\compiletype}{print}}
	{
		\def\printmode{1}
	}{}

\setbeamercovered{invisible}

%\usepackage[citestyle=authoryear,bibstyle=numeric,hyperref,backend=biber]{biblatex}
%\addbibresource{templates/example.bib}
%\bibhang1em

\begin{document}
	
\selectlanguage{ngerman}


%title page
\begin{frame}
	\titlepage
\end{frame}

%table of contents
\ifdefined\printmode
	\ifdefined\compileall \else
	\begin{frame}{Gliederung}
		\tableofcontents
	\end{frame}
\fi\fi

\fi

%TODO POP Quiz

\section{Vollständige Induktion}
\begin{frame} {Was ist überhaupt vollständige Induktion?}
	\begin{itemize}
		\item Beweisverfahren
		\item In der Regel zu zeigen: Eine Aussage gilt für alle $n \in \mathbb{N}_+$, manchmal auch für alle $n \in \mathbb{N}_0$
		\item Man schließt "induktiv" von einem n auf n+1
		\item Idee: Wenn die Behauptung für ein beliebiges \textbf{festes} n gilt, dann gilt sie auch für den Nachfolger n+1 (und somit auch für dessen Nachfolger und schließlich für alle n)
	\end{itemize}
\end{frame}

\begin{frame}{Struktur des Beweises}
	Behauptung: (\textit {kurz} \textbf{Beh.:})\\
	Beweis: (\textit{kurz} \textbf{Bew.:})
	\begin{itemize}
		\pause
		\item Induktionsanfang: (\textit{kurz} \textbf{IA:})
		\begin{itemize}
			\item Zeigen, dass Behauptung für Anfangswert gilt (oft $n=1$)
			\item Auch mehrere (z.B. zwei) Anfangswerte möglich
		\end{itemize}
		\pause
		\item Induktionsvoraussetzung: (\textit{kurz} \textbf{IV:})
		\begin{itemize}
			\item Sei $n \in \mathbb{N}_+$ (bzw. $n \in \mathbb{N}_0$) fest aber beliebig und es gelte [Behauptung einsetzen]
		\end{itemize}
		\pause
		\item Induktionsschritt: (\textit{kurz} \textbf{IS:})
		\begin{itemize}
			\item Behauptung für n+1 auf n zurückführen
			\item Wenn induktive Definition gegeben: verwenden!
			\item Sonst: Versuche Ausdruck, in dem (n+1) vorkommt umzuformen in einen Ausdruck, in dem nur n vorkommt
		\end{itemize}
	\end{itemize}
\end{frame}
%Zwei Beispiele, Lösungen siehe gleicher Ordner
% https://www.cl.cam.ac.uk/~mgk25/kuhn-fa.pdf
%Einfacheres S.8
\begin{frame}
	\textbf{Aufgabe}\\
	\begin{eqnarray*}
		&x_0 := 0\\
		&\text{Für alle } n \in \mathbb{N}_0: x_{n+1} := x_n + 2n +1
	\end{eqnarray*}			
	\textit{Zeige mithilfe vollständiger Induktion, dass für alle} $n \in \mathbb{N}_0$ \\
	\begin{center}$x_n = n^2$\end{center}
	gilt.
\end{frame}

\begin{frame}{Formale Sprache}
	\begin{itemize}
		\pitem Was war nochmal $A^*$? Menge aller Wörter \emph{beliebiger} Länge über Alphabet $A$.
		\pitem Was war nochmal eine formale Sprache?
	\end{itemize}
	
	\pause
	
	\begin{block}{Formale Sprache}
		Eine Formale Sprache $L$ \pause über einem Alphabet $A$ ist eine Teilmenge $L \subseteq A$.
	\end{block}

	\pause Als Beispiel von vorigen Folien:
	
	\begin{itemize}
		\pitem $A := \{b, n, a\}$.
		\begin{itemize}
			\pitem $L_1 := \{ban, baan, nba, aa\}$ ist eine mögliche formale Sprache über $A$.
			\pitem $L_2 := \{banana, bananana, banananana, ...\}$ \pause $ = \{w : w = bana(na)^k, k \in \N\}$ auch.
			\pitem $L_3 := \{ban, baan, baaan, ...\}$ auch. \pause Andere Schreibweise? \pause \\ $ L_3 = \{w : w = ba^kn, k \in \N \}$
		\end{itemize}
	\end{itemize}
\end{frame}

\begin{frame}{Produkt von Sprachen}
	\begin{block}{Produkt von formalen Sprachen}
		Von zwei formalen Sprachen $L_1, L_2$ \pause lässt sich das Produkt \pause $L_1 \cdot L_2$ \pause bilden mit \pause $L_1 \cdot L_2 = \{w_1w_2 : w_1 \in L_1 $ und $w_2 \in L_2 \}$.
	\end{block}

	Sei $A := \{a, b\}, B := \{A, B, C, D, E, F\}$.
	
	\begin{itemize}
		\pitem Sprache $L_1 \subseteq A$, die zuerst drei $a$'s enthält und dann beliebig viele $b$'s? $L_1 = \{aaa\}\cdot\{b\}^*$.
		\pitem Sprache $L_2 \subseteq A$, die das Teilwort $ab$ nicht enthält? \pause $L_2 = \{b\}^*\{a\}^*$.
		\pitem Sprache $L_3 \subseteq A$, die alle Wörter über $A$ enthält außer $\epsilon$? \pause $L_3 = A \cdot A^* \pause = A^* \backslash \{\epsilon\}$.
		\pitem Sprache $L_4$, die alle erlaubten Java Variablennamen enthält.
		\begin{itemize}
			\pitem $B := \{_,a,b,...,z,A,B,...,Z\}$
			\pitem $C := B \cup \Z_9$
			\pitem $L_4 \subseteq C \pause = (B \cdot C^*) \backslash \{if, class, while, ...\}$
		\end{itemize}
	\end{itemize}
\end{frame}

\begin{frame}{Produkt von Sprachen}
	(Exkurs zur Linearen Algebra)\pause
	
	\begin{taskblock}{Übung zu Produkt von formalen Sprachen}
		Sei $A$ ein beliebiges Alphabet und $M := \{L : L $ ist formale Sprache über $A \} \pause = 2^A$. \pause Produkt von Sprachen lässt sich auch als Abbildung bzw. Verknüpfung $\cdot : M \times M \rightarrow M$ darstellen.
		
		Zeige: 
		\begin{itemize}
			\pitem Die Verknüpfung $\cdot$ ist assoziativ.
			\pitem Es gibt (mindestens) ein Element $e \in M$, sodass für alle $x \in M$ gilt: $x \cdot e = e \cdot x = x$.
			\pitem Für jedes $x \in M$ gibt es (mindestens) ein Element $y \in M$, sodass gilt: $x \cdot y = y \cdot x = \hat{e} \in M$.
		\end{itemize}
	\end{taskblock}
\end{frame}

\begin{frame}{Produkt von Sprachen}
	\pause 
	
	Seien $L_1, L_2, L_3 \in M$.
	
	\begin{itemize}
		\item Die Verknüpfung $\cdot$ ist assoziativ:
		\begin{itemize}
			\pitem $(L_1 \cdot L_2) \cdot L_3 \pause = (\{w_1\cdot w_2 : w_1 \in L_1, w_2 \in L_2\}) \cdot L_3 \pause = \{w_1w_2w_3 : w_1 \in L_1, w_2 \in L_2, w_3 \in L_3\} \pause = L_1 \cdot (\{w_2w_3 : w_2 \in L_2, w_3 \in L_3\}) \pause = L_1 \cdot (L_2 \cdot L_3)$.
		\end{itemize}
	
		\pitem Es gibt (mindestens) ein Element $e \in M$, sodass für alle $x \in M$ gilt: $x \cdot e = e \cdot x = x$.
		\begin{itemize}
			\pitem $e := \{\epsilon\}$.
			\pitem $L_1 \cdot \{\epsilon\} \pause = L_1 \pause = \{\epsilon\} \cdot L_1$
		\end{itemize}
	
		\pitem Für jedes $x \in M$ gibt es (mindestens) ein Element $y \in M$, sodass gilt: $x \cdot y = y \cdot x = \hat{e} \in M$.
		\begin{itemize}
			\pitem $y := \emptyset \pause =: \hat{e}$
			\pitem $L_1 \cdot \emptyset = \emptyset = \hat{e} = \emptyset = \emptyset \cdot L_1$
		\end{itemize}
	\end{itemize}

	Ist damit $(M, \cdot)$ eine Gruppe? \pause Leider nicht. \pause Mussten bei der letzten Aufgabe etwas tricksen, $(M, \cdot)$ wäre eine Gruppe wenn $e = \hat{e}$\pause , aber $e = \{\epsilon\} \neq \hat{e} = \emptyset$.
\end{frame}

\begin{frame}{Potenz von Sprachen}
	
	\begin{block}{Potenz von Sprachen}
		Potenz von formellen Sprachen ist wie folgt definiert:
		\begin{itemize}
			\pitem $L^0 := \{\epsilon\}$
			\pitem $L^{i+1} := L^i \cdot L$ für $i \in \N_+$.
		\end{itemize}
	\end{block}

	\begin{itemize}
		\pitem $L_1 := \{a\}$.
		\begin{itemize}
			\pitem $L_1^0 = \{\epsilon\}$. \pause $L_1^1 \pause = \{\epsilon\} \cdot L_1 \pause = L_1$.
			\pitem $L_1^2 = (\{\epsilon\} \cdot L_1) \cdot L_1 \pause = \{aa\}$.
		\end{itemize}
		\pitem $L_2 := \{ab\}^3\{c\}^4$
		\begin{itemize}
			\pitem $L_2^0 = \{\epsilon\}, L_1^1 = ...$
			\pitem $L_2^2 \pause = (\{ab\}^3\{c\}^4)^2 \pause = (\{ab\}^3\{cccc\})^2 \pause = \{abababcccc\}^2 \pause = \{abababccccabababcccc\}$.
		\end{itemize}
		\pitem $L_3 := (\{a\} \cup \{b\})^2 \pause = \{aa, ab, ba, bb\}$
	\end{itemize}

\end{frame}

\begin{frame}{Konkatenationsabschluss bei formalen Sprachen}
	\p
	\begin{block}{Konkatenationsabschluss}
		Zu einer formalen Sprache $L$ \pause ist der Konkatenationsabschluss $L^*$ definiert \pause als $L^* := \bigcup\limits_{i \in \N_0} L^i$.
	\end{block}
	\p
	\begin{block}{$\epsilon$-freie Konkatenationsabschluss}
		Zu einer formalen Sprache $L$ \pause ist der $\epsilon$-freie Konkatenationsabschluss $L^+$ definiert \pause als $L^+ := \bigcup\limits_{i \in \N_+} L^i$.
	\end{block}

	\begin{itemize}
		\pitem Warum gilt $\epsilon \notin L^+$ von beliebiger formeller Sprache $L$?
		\pitem $L := \{a, b, c\}. L^* \pause = \{\epsilon, a, aa, ab, ac, aaa, aab, \dots, b, ba, bb, \dots \}$
	\end{itemize}
\end{frame}

%TODO Aufgabe zu konkatenationsabschluss, siehe tutor blätter
%TODO Binär und zahlendarstellung
		

\ifdefined\compileall
\else


\ifthenelse{\equal{\compiletype}{print}}
{

\begin{frame}{Informationen}
	
	\begin{columns}
		\begin{column}{0.5\textwidth}
			
			\begin{block}{Zum Tutorium}
				\begin{itemize}
					\item Lukas Bach
					\item Tutorienfolien auf: 
					\begin{itemize}
						\item \url{http://gbi.lukasbach.com}
					\end{itemize}
					\item Tutorium findet statt:
					\begin{itemize}
						\item Donnerstags, 14:00 - 15:30
						\item 50.34 Informatikbau, -107
					\end{itemize}
				\end{itemize}
			\end{block}
			
			\begin{block}{Mehr Material}
				\begin{itemize}
					\item ILIAS der Vorlesung:
					\begin{itemize}
						\item kommt noch. %TODO
					\end{itemize}
					\item Ehemalige GBI Webseite:
					\begin{itemize}
						\item \url{http://gbi.ira.uka.de}
						\item Altklausuren!
					\end{itemize}
				\end{itemize}
			\end{block}
			
		\end{column}
		\begin{column}{0.5\textwidth}
			
			\begin{block}{Zur Veranstaltung}
				\begin{itemize}
					\item Grundbegriffe der Informatik
					\item Klausurtermin:
					\begin{itemize}
						\item 06.03.2017, 11:00
						\item Zwei Stunden Bearbeitungszeit
						\item 6 ECTS für Informatiker und Informationswirte, 4 ECTS für Mathematiker und Physiker
					\end{itemize}
				\end{itemize}
			\end{block}
			
			\begin{block}{Zum Übungsschein}
				\begin{itemize}
					\item Übungsblatt jede Woche
					\item Ab 50\% insgesamt hat man den Übungsschein
					\item Keine Voraussetzung für die Klausur, aber für das Modul
				\end{itemize}
			\end{block}
			
		\end{column}
	\end{columns}
	
\end{frame}

}{}

\ifdefined\livebeamermode
	\begin{frame}
		\includegraphics[width=\linewidth]{images/thatsall.png}
	\end{frame}
\fi

\end{document}

\fi