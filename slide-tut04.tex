\def\tutdate{17.11.2016}
\ifdefined\compileall \else
\ifdefined\compiletype
	\documentclass[handout]{beamer}
\else
	\documentclass{beamer}
	\def\compiletype{livebeamer}
\fi

\usepackage{templates/beamerthemekitwide}

\usepackage[utf8]{inputenc}
\usepackage[T1]{fontenc}
\usepackage[ngerman]{babel}
\usepackage{listings}
\usepackage{hyperref}
\usepackage{graphicx}

\usepackage{amsmath}
\usepackage{amsthm}
\usepackage{amssymb}
\usepackage{polynom}

\usepackage{ifthen}
\usepackage{adjustbox} % for \adjincludegraphics

\usepackage{tikz}
\usepackage{listings}

\usepackage[]{algorithm2e}

\usepackage{colortbl}
\usepackage{verbatim}
\usepackage{alltt}
\usepackage{changes}

\usepackage{pdfpages}
\usepackage{tabularx}

\usepackage{euler}

\newcommand{\markBlue}[1]{\textcolor{kit-blue100}{#1}}
\newcommand{\markGreen}[1]{\textcolor{kit-green100}{#1}}
\newcommand{\vertspace}{\vspace{.2cm}}

%\newcommand{\#}{\markBlue{#1}}

%\newcommand{\pitem}{\pause\item}
\newcommand{\p}{\pause}

% -- MATH MACROS
\newcommand{\thistheoremname}{}
\newcommand{\G}{\mathbb{Z}}
\newcommand{\B}{\mathbb{B}}
\newcommand{\R}{\mathbb{R}}
\newcommand{\N}{\mathbb{N}}
\newcommand{\Q}{\mathbb{Q}}
\newcommand{\C}{\mathbb{C}}
\newcommand{\Z}{\mathbb{Z}}
\newcommand{\F}{\mathbb{F}}
\newcommand{\mi}{\mathrm{i}}
\renewcommand{\epsilon}{\varepsilon}


\newenvironment<>{taskblock}[1]{%
	\setbeamercolor{block title}{fg=kit-orange15,bg=kit-orange100}
	\setbeamercolor{block body}{fg=black,bg=kit-orange30}%
	\begin{block}#2{#1}}{\end{block}}

\setbeamertemplate{enumerate items}[default]

% Aussagenlogik Symbole
\newcommand{\W}{w}
\renewcommand{\F}{f}

% Kodierung
\newcommand{\frepr}{\textbf{repr}}
\newcommand{\fRepr}{\textbf{Repr}}
\newcommand{\fZkpl}{\textbf{Zkpl}}
\newcommand{\fbin}{\textbf{bin}}
\newcommand{\fdiv}{\textbf{ div }}
\newcommand{\fmod}{\textbf{ mod }}

% Speicherabbild
\newenvironment{memory}{\begin{tabular}{r | l}Adresse&Wert\\\hline\hline}{\end{tabular}}
\newcommand{\memrow}[2]{#1 & #2 \\\hline}

% Praedikatenlogik
\newcommand{\objequiv}{\stackrel{\cdot}{=}}
\newcommand{\pval}{val_{D,I,\beta}}

% Neue Befehle
\newcommand{\ip}{\pause} % inline pause, für mitten im satz
\newcommand{\pitem}{\pause\item} % für aufzählungen
\newcommand{\bp}{\pause} % block pause, für zwischen blöcken

\title[Grundbegriffe der Informatik]{Grundbegriffe der Informatik\\Tutorium 33}
\subtitle{}
\author{Lukas Bach, lukas.bach@student.kit.edu}
\date{\tutdate}

\institute{}

\titlelogo{lukasbach}
\titleimage{bg}



\ifthenelse{\equal{\compiletype}{livebeamer}}
	{
		\def\livebeamermode{1}
	}{}

\ifthenelse{\equal{\compiletype}{print}}
	{
		\def\printmode{1}
	}{}

\setbeamercovered{invisible}

%\usepackage[citestyle=authoryear,bibstyle=numeric,hyperref,backend=biber]{biblatex}
%\addbibresource{templates/example.bib}
%\bibhang1em

\begin{document}
	
\selectlanguage{ngerman}


%title page
\begin{frame}
	\titlepage
\end{frame}

%table of contents
\ifdefined\printmode
	\ifdefined\compileall \else
	\begin{frame}{Gliederung}
		\tableofcontents
	\end{frame}
\fi\fi

\fi

%TODO POP Quiz

\section{Vollständige Induktion}
\begin{frame} {Was ist überhaupt vollständige Induktion?}
	\begin{itemize}
		\item Beweisverfahren
		\item In der Regel zu zeigen: Eine Aussage gilt für alle $n \in \mathbb{N}_+$, manchmal auch für alle $n \in \mathbb{N}_0$
		\item Man schließt ``induktiv'' von einem n auf n+1
		\item Idee: Wenn die Behauptung für ein beliebiges \textbf{festes} n gilt, dann gilt sie auch für den Nachfolger n+1 (und somit auch für dessen Nachfolger und schließlich für alle n)
	\end{itemize}
\end{frame}

\begin{frame}{Struktur des Beweises}
	Behauptung: (\textit {kurz} \textbf{Beh.:})\\
	Beweis: (\textit{kurz} \textbf{Bew.:})
	\begin{itemize}
		\pause
		\item Induktionsanfang: (\textit{kurz} \textbf{IA:})
		\begin{itemize}
			\item Zeigen, dass Behauptung für Anfangswert gilt (oft $n=1$)
			\item Auch mehrere (z.B. zwei) Anfangswerte möglich
		\end{itemize}
		\pause
		\item Induktionsvoraussetzung: (\textit{kurz} \textbf{IV:})
		\begin{itemize}
			\item Sei $n \in \mathbb{N}_+$ (bzw. $n \in \mathbb{N}_0$) fest aber beliebig und es gelte [Behauptung einsetzen]
		\end{itemize}
		\pause
		\item Induktionsschritt: (\textit{kurz} \textbf{IS:})
		\begin{itemize}
			\item Behauptung für n+1 auf n zurückführen
			\item Wenn induktive Definition gegeben: verwenden!
			\item Sonst: Versuche Ausdruck, in dem (n+1) vorkommt umzuformen in einen Ausdruck, in dem nur n vorkommt
		\end{itemize}
	\end{itemize}
\end{frame}
%Zwei Beispiele, Lösungen siehe gleicher Ordner
% https://www.cl.cam.ac.uk/~mgk25/kuhn-fa.pdf
%Einfacheres S.8
\begin{frame}
	\textbf{Aufgabe}\\
	\begin{eqnarray*}
		&x_0 := 0\\
		&\text{Für alle } n \in \mathbb{N}_0: x_{n+1} := x_n + 2n +1
	\end{eqnarray*}			
	\textit{Zeige mithilfe vollständiger Induktion, dass für alle} $n \in \mathbb{N}_0$ \\
	\begin{center}$x_n = n^2$\end{center}
	gilt.
\end{frame}

\section{Formale Sprache}

\begin{frame}{Formale Sprache}
	\begin{itemize}
		\pitem Was war nochmal $A^*$? Menge aller Wörter \emph{beliebiger} Länge über Alphabet $A$.
		\pitem Was war nochmal eine formale Sprache?
	\end{itemize}
	
	\pause
	
	\begin{block}{Formale Sprache}
		Eine Formale Sprache $L$ \pause über einem Alphabet $A$ ist eine Teilmenge $L \subseteq A$.
	\end{block}

	\pause Als Beispiel von vorigen Folien:
	
	\begin{itemize}
		\pitem $A := \{b, n, a\}$.
		\begin{itemize}
			\pitem $L_1 := \{ban, baan, nba, aa\}$ ist eine mögliche formale Sprache über $A$.
			\pitem $L_2 := \{banana, bananana, banananana, ...\}$ \pause $ = \{w : w = bana(na)^k, k \in \N\}$ auch.
			\pitem $L_3 := \{ban, baan, baaan, ...\}$ auch. \pause Andere Schreibweise? \pause \\ $ L_3 = \{w : w = ba^kn, k \in \N \}$
		\end{itemize}
	\end{itemize}
\end{frame}

\begin{frame}{Produkt von Sprachen}
	\begin{block}{Produkt von formalen Sprachen}
		Von zwei formalen Sprachen $L_1, L_2$ \pause lässt sich das Produkt \pause $L_1 \cdot L_2$ \pause bilden mit \pause $L_1 \cdot L_2 = \{w_1w_2 : w_1 \in L_1 $ und $w_2 \in L_2 \}$.
	\end{block}

	Sei $A := \{a, b\}, B := \{A, B, C, D, E, F\}$.
	
	\begin{itemize}
		\pitem Sprache $L_1 \subseteq A$, die zuerst drei $a$'s enthält und dann beliebig viele $b$'s? $L_1 = \{aaa\}\cdot\{b\}^*$.
		\pitem Sprache $L_2 \subseteq A$, die das Teilwort $ab$ nicht enthält? \pause $L_2 = \{b\}^*\{a\}^*$.
		\pitem Sprache $L_3 \subseteq A$, die alle Wörter über $A$ enthält außer $\epsilon$? \pause $L_3 = A \cdot A^* \pause = A^* \backslash \{\epsilon\}$.
		\pitem Sprache $L_4$, die alle erlaubten Java Variablennamen enthält.
		\begin{itemize}
			\pitem $B := \{\_,a,b,...,z,A,B,...,Z\}$
			\pitem $C := B \cup \Z_9$
			\pitem $L_4 \subseteq C \pause = (B \cdot C^*) \backslash \{if, class, while, ...\}$
		\end{itemize}
	\end{itemize}
\end{frame}

\begin{frame}{Produkt von Sprachen}	
	\begin{taskblock}{Übung zu Produkt von formalen Sprachen}
		Sei $A$ ein beliebiges Alphabet und $M := \{L : L $ ist formale Sprache über $A \} \pause = 2^A$. \pause Produkt von Sprachen lässt sich auch als Abbildung bzw. Verknüpfung $\cdot : M \times M \rightarrow M$ darstellen.
		
		Zeige: 
		\begin{itemize}
			\pitem Die Verknüpfung $\cdot$ ist assoziativ.
			\pitem Es gibt (mindestens) ein Element $e \in M$, sodass für alle $x \in M$ gilt: $x \cdot e = e \cdot x = x$. (Neutrales Element)
			\pitem Es gibt ein Element $o \in M$, sodass für alle $x \in M$ gilt: $x \cdot o = o = o \cdot x$. (Absorbierendes Element)
		\end{itemize}
	\end{taskblock}
\end{frame}

\begin{frame}{Produkt von Sprachen}
	\pause 
	
	Seien $L_1, L_2, L_3 \in M$.
	
	\begin{itemize}
		\item Die Verknüpfung $\cdot$ ist assoziativ:
		\begin{itemize}
			\pitem $(L_1 \cdot L_2) \cdot L_3 \pause = (\{w_1\cdot w_2 : w_1 \in L_1, w_2 \in L_2\}) \cdot L_3 \pause = \{w_1w_2w_3 : w_1 \in L_1, w_2 \in L_2, w_3 \in L_3\} \pause = L_1 \cdot (\{w_2w_3 : w_2 \in L_2, w_3 \in L_3\}) \pause = L_1 \cdot (L_2 \cdot L_3)$.
		\end{itemize}
	
		\pitem Es gibt (mindestens) ein Element $e \in M$, sodass für alle $x \in M$ gilt: $x \cdot e = e \cdot x = x$. (neutrales Element)
		\begin{itemize}
			\pitem $e := \{\epsilon\}$.
			\pitem $L_1 \cdot \{\epsilon\} \pause = L_1 \pause = \{\epsilon\} \cdot L_1$
		\end{itemize}
	
		\pitem Es gibt ein Element $o \in M$, sodass für alle $x \in M$ gilt: $x \cdot o = o = o \cdot x$. (Absorbierendes Element)
		\begin{itemize}
			\pitem $o := \emptyset$
			\pitem $L_1 \cdot \emptyset = \emptyset = \emptyset \cdot L_1$
		\end{itemize}
	\end{itemize}

	$(M, \cdot)$ ist damit keine Gruppe\p , es existieren keine Invers-Element.
\end{frame}

\begin{frame}{Potenz von Sprachen}
	
	\begin{block}{Potenz von Sprachen}
		Potenz von formellen Sprachen ist wie folgt definiert:
		\begin{itemize}
			\pitem $L^0 := \{\epsilon\}$
			\pitem $L^{i+1} := L^i \cdot L$ für $i \in \N_+$.
		\end{itemize}
	\end{block}

	\begin{itemize}
		\pitem $L_1 := \{a\}$.
		\begin{itemize}
			\pitem $L_1^0 = \{\epsilon\}$. \pause $L_1^1 \pause = \{\epsilon\} \cdot L_1 \pause = L_1$.
			\pitem $L_1^2 = (\{\epsilon\} \cdot L_1) \cdot L_1 \pause = \{aa\}$.
		\end{itemize}
		\pitem $L_2 := \{ab\}^3\{c\}^4$
		\begin{itemize}
			\pitem $L_2^0 = \{\epsilon\}, L_2^1 = ...$
			\pitem $L_2^2 \pause = (\{ab\}^3\{c\}^4)^2 \pause = (\{ab\}^3\{cccc\})^2 \pause = \{abababcccc\}^2 \pause = \{abababccccabababcccc\}$.
		\end{itemize}
		\pitem $L_3 := (\{a\} \cup \{b\})^2 \pause = \{aa, ab, ba, bb\}$
	\end{itemize}

\end{frame}

\begin{frame}{Konkatenationsabschluss bei formalen Sprachen}
	\p
	\begin{block}{Konkatenationsabschluss}
		Zu einer formalen Sprache $L$ \pause ist der Konkatenationsabschluss $L^*$ definiert \pause als $L^* := \bigcup\limits_{i \in \N_0} L^i$.
	\end{block}
	\p
	\begin{block}{$\epsilon$-freie Konkatenationsabschluss}
		Zu einer formalen Sprache $L$ \pause ist der $\epsilon$-freie Konkatenationsabschluss $L^+$ definiert \pause als $L^+ := \bigcup\limits_{i \in \N_+} L^i$.
	\end{block}

	\begin{itemize}
		\pitem Warum gilt $\epsilon \notin L^+$ von beliebiger formeller Sprache $L$?
		\pitem $L := \{a, b, c\}.  L^* \pause = \{\epsilon, a, aa, ab, ac, aaa, aab, \dots, b, ba, bb, \dots \}$
		\pitem $L := \{aa, bc\}.  L^* \pause = \{\epsilon, aa, bc, aa\cdot aa, aa\cdot bc, bc \cdot aa, bc \cdot bc, aa \cdot aa \cdot aa, \dots \}$
	\end{itemize}
\end{frame}

\begin{frame}{Übung zu Konkatenationsabschluss}
	\pause Sei $L := \{a\}^*\{b\}^*$.
	\begin{itemize}
		\pitem Was ist alles in $L$ drin?
		\begin{itemize}
			\pitem $aaabbabbaaabba$? \pause Nein.
			\pitem $aaabb$, $abbaaabba$? \pause Ja, nein.
			\pitem $aaabb$, $abb$, $aaabba$? \pause Ja, ja, nein.
			\pitem $aaabb$, $abb$, $aaabb$, $a$? \pause Alles drin.
		\end{itemize}
		\pitem Was ist alles in $L^*$ drin?
		\begin{itemize}
			\pitem $aaabbabbaaabba$? \pause Ja.
			\pitem $aaabb$, $abbaaabba$? \pause Ja.
			\pitem $aaabb$, $abb$, $aaabba$? \pause Ja.
			\pitem $aaabb$, $abb$, $aaabb$, $a$? \pause Ja.
			\pitem Alle Wörter aus $\{a,b\}^*$! \pause $\rightarrow L^* = \{a,b\}^*$.
		\end{itemize}
	\end{itemize}
\end{frame}

\begin{frame}{Übung zu Konkatenationsabschluss}
	\begin{exampleblock}{Erinnerung}
		\begin{center}
			$L^* := \bigcup\limits_{i \in \N_0} L^i$\qquad
			$L^+ := \bigcup\limits_{i \in \N_+} L^i$
		\end{center}
	\end{exampleblock}

	\begin{taskblock}{Beweisaufgabe}
		Beweise: $L^* \cdot L = L^+$.
	\end{taskblock}

	\pause
	\begin{columns}
		\begin{column}{0.4\textwidth}
			$\subseteq$:
			
			\p\markBlue{Vorraussetzung:} \p $w \in L^* \cdot L$ mit $w = w'w'', w' \in L^*$ und $w'' \in L$.
			
			\vspace{.3cm}
			
			\p Dann existiert ein $i \in \N_0$ mit $w' \in L^i$\p , also $w = w'w'' \in L^i \cdot L \p = L^{i+1}$.
			
			\vspace{.3cm}
			
			\p Weil $i+1\in \N_+$\p , gilt: $L^{i+1} \subseteq L^+$\p , also $w \in L^+$.
		\end{column}
		
		\begin{column}{0.6\textwidth}
			$\supseteq$:
			
			\p\markBlue{Vorraussetzung:} \p $w \in L^*\cdot L$.
			
			\vspace{.3cm}
			
			\p Dann existiert ein $i \in \N_+$ mit $w \in L^i$. \p Da $i \in \N_+$\p , existiert ein $j \in \N_0$ mit $i = j+1$\p , also für ein solches $j \in \N_0$\p : $w \in L^{j+1} \p = L^j \cdot L$.
			
			\vspace{.3cm}
			
			\p Also $w = w'w''$ mit $w' \in L^j$ und $w'' \in L$.
			
			\vspace{.3cm}
			
			\p Wegen $L^j \subseteq L^*$ \p ist $w = w'w'' \p \in L^* \cdot L$.
		\end{column}
	\end{columns}
\end{frame}

\begin{frame}{Übung zu formalen Sprachen}
	$L_1, L_2$ seien formale Sprachen.
	\begin{itemize}
		\pitem Wie sieht $L_1 \cdot L_2$ aus?
		\pitem Wie sieht $L_1^3$ aus?
		\pitem Wie sieht $L_1^2 \cdot L_2 \cdot L_2^0 \cdot L_1^*$ aus?
		\pitem Wie sieht $(L_1^*)^0 \cdot L_2^+$ aus?
	\end{itemize}	
\end{frame}



\section{Übersetzung und Kodierung}

\begin{frame}{Herführung zu Zahlendarstellungen}
	\pause Wir betrachten die Alphabete $A_{dez} := \Z_{10}, A_{bin} := \{0, 1\}, A_{oct} := \Z_8$.
	\begin{itemize}
		\pitem Was können wir daraus machen?
		\pitem $A_{dez}^* \supset \{42, 1337, 999\}$.
		\pitem $A_{bin}^* \supset \{101010, 10100111001, 1111100111\}$.
		\pitem $A_{oct}^* \supset \{52, 2471, 1747\}$.
		\pitem Wir suchen eine Möglichkeit, diese \markGreen{Zahlen} zu \markGreen{deuten}.
		\pitem Aber irgendwie so, dass $42_{\in A_{dez}} \stackrel{Deutung}{=} 101010_{\in A_{bin}} \stackrel{Deutung}{=} 52_{\in A_{oct}}$.
	\end{itemize}
\end{frame}

\begin{frame}{Definition von Zahlendarstellungen}
	\pause
	
	\begin{block}{$Num_k$}
		Einer Zeichenkette $Z_k$ aus Ziffern \p wird mit $Num_k$ eine eindeutige Zahl zugeordnet:
		
		\vspace{.2cm}
		
		\vspace{.2cm}
		
		 \p $Num_k(\epsilon) = 0$
		
		\vspace{.2cm}
		
		 \p $Num_k(wx) = k \cdot Num_k(w) + num_k(x)$ mit $w \in Z_k^*$ und $x \in Z_k$.
	\end{block}

	\pause
	
	\begin{block}{$num_k$}
		Einer einzelnen Ziffer $x \in Z_k$ aus einem Alphabet von Ziffern $Z_k$ wird mit $num_k(x)$ der Wert der Zahl zugewiesen.
	\end{block}

	\pause
	
	\begin{itemize}
		\item Wichtig: $Num_k(w) \neq num_k(w)$!
		\pitem Was ist: $num_{10}(3) \p = 3\p , num_{10}(7) \p = 7\p , num_{10}(11) = \p $ nicht definiert.
		\pitem Für Zahlen $\geq k$: Benutze $Num_k$!
	\end{itemize}
\end{frame}

\begin{frame}{Beispiel zu Zahlendarstellungen}
	$Num_k(\epsilon) = 0$.
	
	$Num_k(wx) = k \cdot Num_k(w) + num_k(x)$ mit $w \in Z_k^*$ und $x \in Z_k$.
	
	\vspace{.3cm}
	
	\p Was ist $Num_{10}(123)$?
	\begin{itemize}
		\pitem $Num_{10}(123) \p = 10 \cdot Num_{10} (12) + num_{10}(3) \p = 10 \cdot ( Num_{10} (1) + num_{10}(2)) + num_{10}(3) \p = 10 \cdot ( num_{10}(1) + 10 \cdot num_{10}(2) ) + num_{10}(3) \p = 10 \cdot ( 1 + 10 \cdot 2 ) + 3 \p = 123$.
	\end{itemize}
	\p Yay?
	
	\p Was ist der dezimale Zahlenwert der Binärzahl 1010? \p Diesmal Basis $k = 2$.
	\begin{itemize}
		\pitem $Num_{2}(1010) \p = 2 \cdot Num_2(101) + num_2(0) \p = 2 \cdot (2 \cdot Num_2(10) + num_2(1) + num_2(0) \p = 2 \cdot (2 \cdot (2 \cdot Num_2(1) + num_2(0) ) + num_2(1) ) + num_2(0) \p = 2 \cdot (2 \cdot (2 \cdot num_2(1) + num_2(0) ) + num_2(1) ) + num_2(0) \p = 2 \cdot ( 2 \cdot (2 \cdot 1 + 0) + 1) + 0) = 10$.
	\end{itemize}
	\p Yay!
\end{frame}

\begin{frame}{Aufgaben zu Zahlendarstellungen}
	$Num_k(\epsilon) = 0$.
	
	$Num_k(wx) = k \cdot Num_k(w) + num_k(x)$ mit $w \in Z_k^*$ und $x \in Z_k$.
	
	\begin{taskblock}{Übungen zu Zahlendarstellungen}
		Berechne den numerischen Wert der folgenden Zahlen anderer Zahlensysteme nach dem vorgestellten Schema:
		\begin{itemize}
			\item $Num_8(345)$.
			\item $Num_2(11001)$.
			\item $Num_2(1000)$.
			\item $Num_4(123)$.
			\item $Num_{16}(4DF)$. (Zusatz)
		\end{itemize}
	\end{taskblock}

	Anmerkung: Hexadezimalzahlen sind zur Basis $16$ und verwenden als Ziffern (in aufsteigender Reihenfolge: $1, 2, 3, 4, 5, 6, 7, 8, 9, A, B, C, D, E, F$.
\end{frame}

\begin{frame}{Aufgaben zu Zahlendarstellungen}
	\pause Lösungen:
	\begin{itemize}
		\pitem $Num_8(345) \p = 229$.
		\pitem $Num_2(11001) \p = 25$.
		\pitem $Num_2(1000) \p = 8$.
		\pitem $Num_4(123) \p = 27$.
		\pitem $Num_{16}(4DF) \p = 1247$. 
	\end{itemize}
\end{frame}

\begin{frame}{Einfachere Umrechnung von Zahlendarstellungen}
	Es gilt: $2(2(2(2(2 \cdot 1 + 0)+1)+0)+1)+0 \p = 2^4 \cdot 1 + 2^4 \cdot 0 + 2^3 \cdot 1 + 2^2 \cdot 0 + 2^1 \cdot 1 + 2^0 \cdot 0$.
	
	\p Daher, einfachere Rechenweise: $Num_k(w) = k^0 \cdot w(0) + k^1 \cdot w(1) + k^2 \cdot w(2) + ...$.
	
	\p Was sind folgende Zahlen in Dezimal im Kopf gerechnet?
	
	\begin{itemize}
		\pitem $Num_2(10101) \p = 21$.
		\pitem $Num_2(11101) \p = 29$.
		\pitem $Num_2(1111111111) \p = 1023$.
	\end{itemize}
	
\end{frame}
		
\begin{frame}{Einfachere Umrechnung von Zahlendarstellungen}
	$Num_k(w) = k^0 \cdot w(0) + k^1 \cdot w(1) + k^2 \cdot w(2) + ...$.
	
	\p Was sind folgende Zahlen in Dezimal im Kopf gerechnet?
	
	\begin{itemize}
		\pitem $Num_{16}(A1) \p = 161$.
		\pitem $Num_{16}(BC) \p = 188$.
		\pitem $Num_{16}(14) \p = 20$.
	\end{itemize}
	
\end{frame}

\begin{frame}{Rechnen mit $div$ und $mod$}
	\pause
	\begin{block}{$div$ Funktion}
		Die Funktion $div$ \markGreen{dividiert ganzzahlig.} \p (Schneidet also den Rest ab).
	\end{block}
	\pause
	\begin{block}{$mod$ Funktion}
		Die Modulo Funktion $mod$ gibt den \markGreen{Rest einer ganzzahligen Division} zurück.
	\end{block}
	\pause
	\begin{itemize}
		\item $22$ div $8 \p = 2 \p $ $(\frac{22}{8} = 2,75)$.
		\pitem $22$ mod $8 \p = 6$.
	\end{itemize}

	\pause Fülle die Tabelle aus:
	
	\begin{tabular}{r | c c c c c c c c c c c c c}
		x & 0 & 1 & 2 & 3 & 4 & 5 & 6 & 7 & 8 & 9 & 10 & 11 & 12\\\hline
		x $div$ 4 & \p 0 & \p 0 & \p 0 & \p 0 & \p 1 & \p 1 & \p 1 & \p 1 & \p 2 & \p 2 & \p 2 & \p 2& \p 3\\
		x $mod$ 4 & \p 0 & \p 1 & \p 2 & \p 3 & \p 0 & \p 1 & \p 2 & \p 3 & \p 0 & \p 1 & \p 2 & \p 3& \p 0\\
	\end{tabular}
\end{frame}

\newcommand{\definitionOfRepr}{
	\begin{align*}
	\fRepr_k(n) =
	\begin{cases}
	\frepr_k(n) & \text{ falls } n < k \\
	\fRepr_k(n \text{ div } k) \cdot \frepr_k(n \text{ mod } k) & \text{ falls } n \geq k
	\end{cases}
	\end{align*}
}

\begin{frame}{Von Zeichen zu Zahlen zurück zu Zahlen}
	$11101_2$ ist also $29_{10}$. \p Was ist $29_{10}$ in binär? \pause
	\begin{block}{$k$-äre Darstellung}
		Die Repräsentation einer Zahl $n$ \p zur Basis $k$ \p lässt sich wie folgt ermitteln:\p
		\definitionOfRepr
		\p Achtung! \p Das $\cdot$ Symbol steht für Konkatenation, nicht für Multiplikation!
	\end{block}
\end{frame}

\begin{frame}{Beispiel zu $Repr_k$}
	\definitionOfRepr
	
	\pause Zum Beispiel: \p 
	\begin{align*}
	\fRepr_2(29) \p &= \fRepr_2(29 \fdiv 2) \cdot \frepr_2(29 \fmod 2)  \\
	\p &= \fRepr_2(14) \cdot \frepr_2(1)\\
	\p &= \fRepr_2(14 \fdiv 2) \cdot \frepr_2(14 \fmod 2) \cdot 1\\
	\p &= \fRepr_2(7) \cdot \frepr_2(0) \cdot 1\\
	\p &= \fRepr_2(7 \fdiv 2) \cdot \frepr_2(7 \fmod 2) \cdot 01\\
	\p &= \fRepr_2(3) \cdot \frepr(1) \cdot 01\\
	\p &= \fRepr_2(3 \fdiv 2) \cdot \frepr(3 \fmod 2) \cdot 101\\
	\p &= \fRepr_2(1) \cdot \frepr(1) \cdot 101\\
	\p &= 11101
	\end{align*}
\end{frame}
\newcommand{\uhd}{_{16}}

\begin{frame}{Beispiel zu $Repr_k$}
\definitionOfRepr

\pause Beispiel mit Hexadezimalzahlen: \p 
\begin{align*}
\fRepr\uhd (29) \p &= \fRepr\uhd (29 \fdiv 16) \cdot \frepr\uhd (29 \fmod 16)  \\
\p &= \fRepr\uhd (1) \cdot \frepr\uhd (13)\\
\p &= 1 \cdot D \p = 1D
\end{align*}
\end{frame}

\begin{frame}{Übung zu $Repr_k$}
	\definitionOfRepr
	\begin{taskblock}{Übung zu $Repr_k$}
		Berechne die Repräsentationen folgender Zahlen in gegebenen Zahlensystemen:
		\begin{itemize}
			\item $\fRepr_2(13)$.
			\item $\fRepr_4(15)$.
			\item $\fRepr\uhd (268)$.
		\end{itemize}
	\end{taskblock}

	\pause Lösungen:
	\begin{itemize}
		\pitem $\fRepr_2(13) \p = 1101$.
		\pitem $\fRepr_4(15) \p = 33$.
		\pitem $\fRepr\uhd (268) \p = 10C$.
	\end{itemize}
\end{frame}

% binary with set length


\newcommand{\definitionOfZkpl}{
\begin{align*}
\fZkpl_{\ell}(x) =
\begin{cases}
0 \fbin_{\ell-1}(x) & \text{ falls } x\geq 0 \\
1 \fbin_{\ell-1}(2^{\ell-1} + x) & \text{ falls } x< 0 
\end{cases}
\end{align*}
}

\begin{frame}{Zweierkomplement}
	\p Was ist mit negative Zahlen?
	
	\pause 
	
	\begin{block}{Zweierkomplement Darstellung}
		Die Zweierkomplementdarstellung einer Zahl $x$ \p mit der Länge $\ell$ ist wie folgt definiert:\p
		\definitionOfZkpl		
	\end{block}
\end{frame}

% huffman codierung?

\ifdefined\compileall
\else


\ifthenelse{\equal{\compiletype}{print}}
{

\begin{frame}{Informationen}
	
	\begin{columns}
		\begin{column}{0.5\textwidth}
			
			\begin{block}{Zum Tutorium}
				\begin{itemize}
					\item Lukas Bach
					\item Tutorienfolien auf: 
					\begin{itemize}
						\item \url{http://gbi.lukasbach.com}
					\end{itemize}
					\item Tutorium findet statt:
					\begin{itemize}
						\item Donnerstags, 14:00 - 15:30
						\item 50.34 Informatikbau, -107
					\end{itemize}
				\end{itemize}
			\end{block}
			
			\begin{block}{Mehr Material}
				\begin{itemize}
					\item ILIAS der Vorlesung:
					\begin{itemize}
						\item kommt noch. %TODO
					\end{itemize}
					\item Ehemalige GBI Webseite:
					\begin{itemize}
						\item \url{http://gbi.ira.uka.de}
						\item Altklausuren!
					\end{itemize}
				\end{itemize}
			\end{block}
			
		\end{column}
		\begin{column}{0.5\textwidth}
			
			\begin{block}{Zur Veranstaltung}
				\begin{itemize}
					\item Grundbegriffe der Informatik
					\item Klausurtermin:
					\begin{itemize}
						\item 06.03.2017, 11:00
						\item Zwei Stunden Bearbeitungszeit
						\item 6 ECTS für Informatiker und Informationswirte, 4 ECTS für Mathematiker und Physiker
					\end{itemize}
				\end{itemize}
			\end{block}
			
			\begin{block}{Zum Übungsschein}
				\begin{itemize}
					\item Übungsblatt jede Woche
					\item Ab 50\% insgesamt hat man den Übungsschein
					\item Keine Voraussetzung für die Klausur, aber für das Modul
				\end{itemize}
			\end{block}
			
		\end{column}
	\end{columns}
	
\end{frame}

}{}

\ifdefined\livebeamermode
	\begin{frame}
		\includegraphics[width=\linewidth]{images/thatsall.png}
	\end{frame}
\fi

\end{document}

\fi